\section{Validation setup}
For the validation setup, the team is lucky enough to be allowed access to an industrial area owned by the E. C. Dahls factory here in Trondheim. There we have a pretty flat, open area that resembles the areas used by the competitions we're going to this summer. The area is also free enough from multipaths, thus allowing the use of RTK-GPS for ground truth generation of position and velocity. \\

The location of the cones will then either be measured by hand using a yard stick, or by hand held laser measurement device if this turns out to work well. The team has some ideas for how to use the hand held laser to find the x and y easily, but we won't know if this works until the test period has begun. 

At the test area we will replicate the different track setups that are specified in the formula student rule set. This setup will hopefully be close enough to the real deal to allow us to work about any problems that arise. Unfortunately the car is not ready for testing before the deadline of this project thesis. All the results are therefore from recorded data from last year in the case of the state estimator, and from simulated data in the case of the SLAM system. \\

